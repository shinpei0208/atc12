\section{System Model and Background}
\label{sec:model}

This paper focuses on a system composed of a GPU and a multi-core CPU.
The GPU is managed by the OS and some other pieces of system software
running on the CPU.
The programming launguage and its runtime subsystem provide programmers
with a set of API functions to use the GPU.
Using these API functions, typical GPU applications take the following
steps: (i) allocate space to the device memory, (ii) move data to the
allocated space on the device memory, (iii) launch computation on the
GPU, (iv) move resultant data back to the host memory, and (v) free the
allocated space from the device memory.
This is the most well-known execution model for today's GPU
programming.

As mentioned in Section~\ref{sec:introduction}, while GPUs have received
considerable attension in many application domains, OS support for GPUs
has not yet been well investigated.
What we believe prevents the research community from tackling GPU
resource management problems is a lack of information on GPU
architectures.
However, the Linux open-source community has investigated the details
of NVIDIA GPU architectures recently~\cite{envytools}, and now we are
ready to explore GPU resource management.
In the following, we define our system model based on \textit{Fermi}
architecture~\cite{Fermi}.
It should be noted that the concept of Gdev is not limited to Fermi, but
is applicable to other architectures that are aligned with the following
model.

\textbf{Command:}


\textbf{Channel:}


\textbf{Address Space:}
Each GPU context runs in separate address space through the virtual
memory management unit.
The address space is also associated with a hardware channel to the GPU.

\textbf{DMA Engine:}

\textbf{Compute Core:}

The data movement between the host and device memory via direct memory
access (DMA) and the computation on the GPU are essentially
non-preemptive, which raises a scheduling challenge.

While traditional microprocessor architectures have opened
specifications and their design is compatible across many generations
over decades, 


The system is composed of a generic multi-core processor and a graphics
card.
We particularly consider CUDA as the underlying programming model, but
the concept of RGEM is applicable to a wide class of GPGPU
programming models, such as OpenCL and HMPP.
The software stack consists of a device driver, a compiler, and a
runtime engine.
The compiler generates GPU code and CPU code.
The CPU code contains a program to launch the GPU code onto the GPU via
the runtime engine and device driver.
The GPU code includes at least one kernel, and a set of data used by
each kernel is uploaded before and downloaded after the kernel execution.
In particular, we assume that GPGPU programs use the following
interfaces.

\begin{itemize}
 \item {\sf MemAlloc(size)} allocates device memory space of {\sf size}
       bytes, and returns a pointer to it.
       If there is no enough space left for the requested size, the
       interface call fails.
 \item \vspace{-0.5em} {\sf MemFree(ptr)} frees the device memory space
       pointed to by {\sf ptr}, which must have been allocated through
       the {\sf MemAlloc} interface.
 \item \vspace{-0.5em} {\sf MemCopyUpload(dst\_addr, src\_buf, size)}
       copies data of {\sf size} bytes from a user-space buffer
       specified by {\sf src\_buf} in the host memory to the device
       memory at the address specified by {\sf dst\_addr}.
       This is a blocking call.
 \item \vspace{-0.5em} {\sf MemCopyDownload(dst\_buf, src\_addr, size)}
       copies data of {\sf size} bytes from the device memory at the
       address specified by {\sf src\_addr} to a user-space buffer
       specified by {\sf dst\_buf} in the host memory.
       This is a blocking call.
 \item \vspace{-0.5em} {\sf Launch(kernel, arguments)} launches the
       kernel program specified by {\sf kernel} with the data parameters
       specified by {\sf arguments}, which is already loaded as part of
       the GPU code in the device memory.
       This is a blocking call.
\end{itemize}

We assume that GPGPU tasks have fixed priorities, which are by default
prioritized over other tasks running on the CPU.
They may or may not execute periodically with deadlines.
The CPU scheduler in the operating system (OS) dispatches tasks
based on their priorities, while the device driver dispatches the
requests to access the GPU when received.
RGEM is aimed at scheduling these requests before passed to the device
driver.
We

There are two types of host-device memory copy operations.
Synchronous memory copy operations block host programs on the CPU until
copy is done, and could also block themselves on the GPU if other memory
copy operations or some device programs are still running.
Asynchronous memory copy operations, on the other hand, block neither
host programs on the CPU nor device programs on the GPU.

Launching device programs onto the GPU is a non-blocking operation in
host programs, while copying memory between the host and device is
either syncmemorycopy operations.
The host programs that have launched device programs are blocked only
when they explicitly call synchronization functions.

Multiple device programs are allowed to execute simultaneously on the
GPU, if they belong to the same context.

\textbf{Limitations:}