\begin{abstract}
 The graphics processing unit (GPU) has become a very powerful platform
 embracing a concept of heterogeneous many-core computing.
 Despite its significant benefits in performance, however, the
 application domains of GPUs are currently limited, largely due to
 a lack of first-class resource management primitives to support the GPU
 in general-purpose time-sharing systems.

 We present Gdev, a new approach to GPU resource management in the
 operating system (OS), which allows user-space applications and the OS
 itself to use the GPU as first-class computing resources.
 It integrates runtime support into the OS, coordinated with the
 device driver, to extend a class of applications that can benefit from
 the GPU.
 This runtime-unified OS approach also enhances the memory management
 and scheduling capabilities for the GPU.
 Specifically, Gdev supports shared memory for inter-process
 communication among GPU contexts and memory swapping for memory
 allocation demands that exceed the physical space. 
 Scheduling is also provided to control GPU resource usage for
 computations and data transmissions.
 Furthermore, Gdev enables the GPU to be virtualized into logical GPUs,
 isolating a certain potion of GPU resources from other time-sharing users.

 We implement an open-source prototype of Gdev for Linux on NVIDIA
 GPUs, and identify the advantage and disadvantage of using Gdev,
 compared to proprietary software and previous work.
 Our detailed experiments show that Gdev can, for instance, gain about
 2x speedups for OS filesystem encryption by using the GPU, improve the
 makespans of data-flow programs by up to 50\%, and maintain virtual
 GPUs utilization within an error of 10\%.

\end{abstract}