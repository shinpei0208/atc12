\begin{abstract}
 The graphics processing unit (GPU) has become a very powerful platform
 embracing a concept of heterogeneous many-core computing.
 Despite its significant benefits in performance, however, the
 application domains of GPUs are currently limited, largely due to
 a lack of resource management primitives to support GPUs in
 time-sharing general-purpose environments.

 In this paper, we present Gdev, a new approach to GPU resource
 management in the operating system (OS), which enables user-space
 applications and the OS itself to use GPUs as first-class computing
 resources.
 Gdev employs the runtime library inside the OS, coordinated
 with the device driver to extend a class of applications that
 benefit from GPUs. This runtime-unified approach also enhances the
 memory management and scheduling of GPU applications.
 Specifically, Gdev supports shared memory for inter-process
 communication among GPU contexts and data allocation beyond the
 physical device memory space.
 Scheduling is also provided to control GPU resource usage with regards
 to computations and data transmissions.
 Gdev further enables the GPU to be virtualized into logical GPUs,
 isolating a certain potion of the GPU resource from other
 time-sharing users.

 We implement an open-source prototype of Gdev for Linux on NVIDIA
 GPUs, and identify the advantage and disadvantage of using Gdev,
 compared to proprietary software and previous work.
 Our detailed experiments show that Gdev can, for instance, gain about
 2x speedups for OS filesystem encryption by using the GPU, improve the
 makespans of data-flow programs by up to 50\% over proprietary
 software, and maintain the desired utilization of virtual GPUs within
 errors of 10\%.
 %While some performance loss is observed in running a standalone
 %user-space program, 

 
\end{abstract}