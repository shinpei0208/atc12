\section{Related Work}
\label{related_work}

\textbf{GPU resource management:}
The GPU is a compute device that essentially requires resource management
to operate.
Recently the research community has developed novel approaches GPU
resource management using OS, runtime library, compiler, and/or
application solutions.

TimeGraph~\cite{Kato_ATC11} and GERM~\cite{Bautin_MCNC08} are GPU
command-driven schedulers integrated in the device driver.
Specifically, TimeGraph provides priority and resource reservation
support for GPU applications, extended with resource sharing
protocols~\cite{Kato_RTAS11}, while GERM addresses fair-share
resource allocation. 
Gdev supports similar scheduling capabilities, but is based on an
API-driven model, where a scheduler is invoked only when tasks use
computing resources on the GPU, eliminating runtime overhead.

PTask~\cite{Rossbach_SOSP11} is an OS abstraction for GPU applications,
which minimizes data movement between the host and device memory through
a data-flow programming model, and also addresses scheduling problems.
CGCM~\cite{Jablin_PLDI11} is a compiler and runtime library solution to
dynamically and automatically optimize host-device data communication,
simiar to PTask.
Gdev does not support such data-flow programming or automatic code
generation; however, it provides programmers with a first-class IPC
primitive for shared memory that can reduce data movement overhead when
communicating with different contexts.

RGEM~\cite{Kato_RTSS11} is a user-space runtime model for real-time
GPGPU applications.
It adovocates split transactions for host-device data transmissions
to bound blocking times, and provides separate queues to demultiplex
schedulers for data transactions and computations.
Although Gdev equips a similar design of separate queues, it
addresses a core challenge of unifying resource management
and runtime support into the OS in order to overcome fundamental
user-space limitations.

Comparisons of Gdev and representatives of the above GPU resource
management approaches are summarized in Table~\ref{tab:related_work}.

\begin{table*}[t]
 \caption{Comparisons of Gdev and prior GPU resource management
 approaches.}
 \label{tab:related_work}
 \begin{center}
  {\sf
  \begin{tabular}{|l|p{12.8cm}|}
   \hline
   \hline
  \end{tabular}
  }
 \end{center}
\vspace{-1em}
\end{table*}


\textbf{GPUs as OS resources:}
One of significant limitations on current GPU programming frameworks is
that GPU applications must reside in the user space.
KGPU~\cite{Sun_SECURITY11_Poster} is a combination of the OS kernel
module and user-space daemon, which allows the OS to use GPUs by
upcalling the user-space daemon from the OS to access the GPU.
On the other hand, Gdev provides OS modules with a set of traditional
API functions for GPU programming, such as CUDA.
Hence, legacy GPU application code can run in the OS without any
modifications and needs not to move back and forth between the user
space and OS.