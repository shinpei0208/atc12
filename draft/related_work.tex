\section{Related Work}
\label{related_work}

\textbf{GPU Resource Management:}
The GPU is a compute device that essentially requires resource management
to operate.
Recently the research community has developed novel approaches GPU
resource management using OS, runtime library, compiler, and/or
application solutions.

TimeGraph~\cite{Kato_ATC11} and GERM~\cite{Bautin_MCNC08} are GPU
command-driven schedulers integrated in the device driver.
Specifically, TimeGraph provides priority and resource reservation
support for GPU applications, extended with resource sharing
protocols~\cite{Kato_RTAS11}, while GERM addresses fair-share
resource allocation. 
Gdev supports similar scheduling capabilities, but is based on an
API-driven model, where a scheduler is invoked only when tasks use
computing resources on the GPU, eliminating runtime overhead.
A command-driven scheduler also fails to control GPU resource usage
precisely, since it cannot recognize what command sequences would launch
computations on the GPU, and hence it accounts for all commands even
though not using GPU computing resources.

PTask~\cite{Rossbach_SOSP11} is an OS abstraction for GPU applications
to minimize data movement between the host and device memory through
a data-flow programming model.
It also addresses scheduling problems.
CGCM~\cite{Jablin_PLDI11} is a compiler and runtime library solution to
dynamically and automatically optimize host-device data communication,
simiar to PTask.
Gdev does not support such data-flow programming or automatic code
generation; however, it provides programmers with a first-class IPC
primitive for shared memory that can reduce data movement overhead in
communicating with different contexts.

RGEM~\cite{Kato_RTSS11} is a user-space runtime model for real-time
GPGPU applications.
It adovocates split transactions for host-device data transmissions
to bound blocking times, and provides separate queues to demultiplex
schedulers for data transactions and computations.
Although Gdev equips a similar design of separate queues, it
addresses a core challenge of unifying resource management
and runtime support into the OS in order to overcome fundamental
user-space limitations.

In addition to the above differences, Gdev provides virtual GPUs that
enable users to view a physical GPU as multiple logical GPUs for
resource usage isolation.
Furthermore, our design and implementation of Gdev newly developed in
this paper are self-contained, allowing the OS to fully control and even
use GPUs as first-class computing resources, whereas the previous work
more or less depend on the proprietary software or existing graphics
drivers, which forces their solutions, if not concepts, to partly adhere
to user-space runtimes.

\begin{comment}
Comparisons of Gdev and representatives of the above GPU resource
management approaches are summarized in Table~\ref{tab:related_work}.
\begin{table*}[t]
 \caption{Comparisons of Gdev and prior GPU resource management
 approaches.}
 \label{tab:related_work}
 \begin{center}
  {\sf
  \begin{tabular}{|l|p{12.8cm}|}
   \hline
   \hline
  \end{tabular}
  }
 \end{center}
\vspace{-1em}
\end{table*}
\end{comment}


\textbf{GPUs as OS Resources:}
A significant limitation on current GPU programming frameworks is
that GPU applications must reside in the user space.
KGPU~\cite{Sun_SECURITY11_Poster} is a combination of the OS kernel
module and user-space daemon, which allows the OS to use GPUs by
upcalling the user-space daemon from the OS to access the GPU.
On the other hand, Gdev provides OS modules with a set of traditional
API functions for GPU programming, such as CUDA.
Hence, legacy GPU application code can run in the OS without any
modifications and needs not to move back and forth between the user
space and OS.

\textbf{GPU Virtualization:}
VMGL~\cite{Lagar-Cavilla_VEE07} virtualizes GPUs at the OpenGL
API level, and VMware's Virtual GPU~\cite{Dowty_SIGOPS09} exhibits I/O
virtualization through graphics runtimes.
On the other hand, Pegasus~\cite{Gupta_ATC11} uses a hypervisor,
Xen~\cite{Barham_SOSP03} in particular, to co-schedule GPUs and virtual
CPUs in VMs.
Nonetheless, these virtualization systems rely on user-space runtimes
provided by proprietary software, excluding GPU resource management.
In addition, they are mainly designed to make GPUs available in
virtualized environtments, but are not tailored to isolate GPU resources
among users.
Gdev provides virtual GPUs with strong time and space partitioning, and
hence could underlie these GPU virtualization systems.

\textbf{I/O Scheduling:}
GPU scheduling deals with a non-preemptive nature of execution as well
as traditional I/O scheduling.
Several disk bandwidth-aware schedulers~\cite{Gulati_FAST09,
Povzner_EUROSYS08, Wang_FAST07}, for example, contain a similar idea to
the Gdev scheduler.
Unlike typical I/O devices, however, GPUs are coprocessors operating
asynchronously with own sets of execution contexts, registers, and memory.
Therefore, Gdev adopts a scheduling algorithm more appropriate for
compute-intensive workload.

\textbf{Compile-Time and Application Approaches:}
GPU resource usage can be also managed in user application
programs~\cite{Chen_IPDPS10,Guevara09,Saba_RTSS11}, but these approaches
essentially require the programs to be modified or recompiled using
specific compilers and algorithms.
Thereby, a generality of programming needs to be compromised.
Under Gdev, on the other hand, applications can use traditional GPU
programming languages.

