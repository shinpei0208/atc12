\begin{abstract}
 The graphics processing unit (GPU) has become a very powerful platform 
 embracing a concept of heterogeneous many-core computing.
 However, application domains of GPUs are currently limited to specific
 systems largely due to a lack of ``first-class'' GPU resource
 management for general-purpose multi-tasking systems.

 We present Gdev, a new ecosystem of GPU resource management in the
 operating system (OS), which allows the user space as well as the OS
 itself to use GPUs as first-class computing resources.
 More specifically, Gdev provides a virtual memory management scheme to
 allow GPU contexts to allocate device memory exceeding the physical
 capacity, and share memory space with other contexts for inter-process
 communication (IPC).
 It also supports a GPU scheduling scheme to virtualize a single GPU
 into multiple logical GPUs, providing an isolation among specific user
 groups in multi-tasking systems.

 Our evaluation using an open-source implementation of Gdev with Linux
 and an NVIDIA GPU shows that Gdev is competitive to proprietary
 software even in basic performance.
 In addition, for instance, we gain about a 2x speedup for an encrypted
 file system by allowing the OS to use the GPU.
 The shared device memory scheme also improves makespans of data-flow
 programs by up to 49\%, and an error in the utilization of virtualized
 GPUs provided by Gdev is limited within 7\%.
\end{abstract}